%!TEX TS-program = xelatex
%!TEX encoding = UTF-8 Unicode
\documentclass[11pt,a4paper]{moderncv}
% moderncv themes
%\moderncvtheme[blue]{casual}                 % optional argument are 'blue' (default), 'orange', 'red', 'green', 'grey' and 'roman' (for roman fonts, instead of sans serif fonts)
\moderncvtheme[blue]{classic}                % idem
\usepackage{xunicode, xltxtra}
\XeTeXlinebreaklocale "zh"
\widowpenalty=10000

%\setmainfont[Mapping=tex-text]{文泉驿正黑}

% character encoding
\usepackage[utf8]{inputenc}                   % replace by the encoding you are using
\usepackage{CJK}

% adjust the page margins
\usepackage[scale=0.8]{geometry}
\recomputelengths                             % required when changes are made to page layout lengths
\setmainfont[Mapping=tex-text]{Hiragino Sans GB}
\setsansfont[Mapping=tex-text]{Hiragino Sans GB}
\CJKtilde

% personal data

%% start of file `template-zh.tex'.
%% Copyright 2006-2012 Xavier Danaux (xdanaux@gmail.com).
%
% This work may be distributed and/or modified under the
% conditions of the LaTeX Project Public License version 1.3c,
% available at http://www.latex-project.org/lppl/.

% 个人信息
\firstname{张}
\familyname{琪}
\title{个人简历}                      % 可选项、如不需要可删除本行
\address{海淀区新街口外大街19号}{100875 北京}             % 可选项、如不需要可删除本行
\mobile{+86~18810500641}                         % 可选项、如不需要可删除本行
%\phone{+2~(345)~678~901}                          % 可选项、如不需要可删除本行
%\fax{+3~(456)~789~012}                            % 可选项、如不需要可删除本行
\email{zhangqi@mail.bnu.edu.cn}                    % 可选项、如不需要可删除本行
%\homepage{dinever.com}                  % 可选项、如不需要可删除本行
\extrainfo{求职意向:数据挖掘、数据分析方向}                  % 可选项、如不需要可删除本行
\photo[64pt]{test.jpg}                  % ‘64pt’是图片必须压缩至的高度、‘0.4pt‘是图片边框的宽度 (如不需要可调节至0pt)、’picture‘ 是图片文件的名字;可选项、如不需要可删除本行
%\quote{引言(可选项)}                           % 可选项、如不需要可删除本行

% 显示索引号;仅用于在简历中使用了引言
%\makeatletter
%\renewcommand*{\bibliographyitemlabel}{\@biblabel{\arabic{enumiv}}}
%\makeatother

% 分类索引
%\usepackage{multibib}
%\newcites{book,misc}{{Books},{Others}}
%----------------------------------------------------------------------------------
%            内容
%----------------------------------------------------------------------------------
\begin{document}
\maketitle

\section{教育背景}
\cventry{2010 -- 2014}{本科}{国际关系学院信息科技系}{信息管理与信息系统}{}{}  % 第3到第6编码可留白
\cventry{2014 -- 2017}{硕士}{北京师范大学系统科学学院}{系统工程}{}{}  % 第3到第6编码可留白

%\section{毕业论文}
%\cvitem{题目}{\emph{题目}}
%\cvitem{导师}{导师}
%\cvitem{说明}{\small 论文简介}

\section{社区}
\cventry{Blog}{\url{https://www.cnblogs.com/zhangqifire/}}{技术博客}{}{}{}
\cventry{GitHub}{\url{http://github.com/zhangqifire}}{}{}{}{}

\section{项目经历}
\renewcommand{\baselinestretch}{1.2}

\cventry{2015--2016}
{鱼群轨迹追踪算法研究}
{Matlab+opencv+c}
{自然基金项目}{}
{鱼群轨迹追踪算法是使用模式识别技术进行目标追踪,包括目标检测,数据关联等技术,我才用了神经网络方法进行目标检测,使用匈牙利算法讲数据关联转化为线性分配问题,从而解决鱼群追踪问题。目前算法能够实现对各种应用环境下遮挡较少的群体进行轨迹追踪。
\\使用统计物理的方法对鱼群实验数据进行分析,研究信息在其中的作用规律,探究鱼群的运动模型
}

\cventry{2014}
{基于AHP和协同过滤算法的电影推荐系统}
{Jsp+Servlet}
{毕业论文}{}
{基于Java和mysql数据库实现的一个电影推荐网站,使用了AHP层次分析法解决协同过滤算法的冷启动问题,实现了基本的电影的个性化推荐。}

\cventry{2012}
{基于android的手机无线点餐系统}
{Java+sqlite}
{团队项目}{}
{本项目在Android2.3手机系统上实现无线点餐,实现了手机端的点餐功能和服务器端的管理查看功能,我主要负责手机端的界面设计和功能实现。本项目参加2012年大学生创新计划获得国际关系学院优秀项目奖}

\section{语言技能}
\cvline{英语}{\textbf{CET-6: 532},擅长读写,经常阅读英文文档、教程,研究生期间经常与外国专家交流。}

\section{专业技能}
\cvline{编程语言}{Python,R,Matlab >   Java  > C#}
\cvline{数据库}{Oracle,MySQL,Sqlserver}
\cvline{工具}{LaTeX}

\section{获得奖励} % (fold)
\cventry{2013}
{高教杯全国大学生数学建模竞赛北京市甲组二等奖}{}{}{}{}
\cventry{2014}
{美国数学建模竞赛二等奖}{}{}{}{主要使用元胞自动机模型模拟交通流,进而评估和改进交通规则使通行效率更高。个人主要负责模型实现部分。}
\cventry{2015}
{全国研究生数学建模竞赛三等奖}{}{}{}{主要使用遗传算法、目标优化等方法对列车行驶方案进行了优化,使列车在行驶过程中的能量消耗最小。}
\cventry{2015}
{2015年北京师范大学研究生学业奖学金一等奖}{}{}{}{}

\section{校园经历} % (fold)
\cventry{2014--2016}
{系统科学学院2014硕士班班长}{}{}{}{从入学以来一直担任所在硕士班班长兼任团支书,多次组织各类班级活动,从活动准备到撰写项目基金申请书和项目总结均能较好的组织完成,任期带领所在班级获得2015年优秀班集体称号和优秀团支部二等奖。}


\section{自我评价} % (fold)
\cventry{}
{熟练使用python、R等语言进行数据挖掘和分析,熟悉常用的机器学习和数据挖掘算法如关联规则算法、Kmeans聚类、神经网络等等}
\cventry{}
{具有较强的数学建模能力,熟练掌握多主体建模、遗传算法、复杂网络的分析方法,能够使用上述方法解决实际问题。}
\closesection{}                   % needed to renewcommands
\renewcommand{\listitemsymbol}{-} % change the symbol for lists

\end{document}


%% 文件结尾 `template-zh.tex'.
